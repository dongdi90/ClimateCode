\documentclass[a4paper,12pt]{article}
\usepackage[utf8x]{inputenc}
\usepackage{graphicx}
\usepackage{subfig}
\usepackage[margin=1in]{geometry}
\usepackage[countmax]{subfloat}
\newcommand{\tab}{\hspace*{2em}}
%opening
\title{Personal Statement NSFGRF}
\author{Matthew Le}

\begin{document}

\maketitle
Pursuing a PhD is something that I never really considered until about a year ago.  It always seemed more practical to just go straight into the workforce after college.  However, after working in two different research labs, I have a very different perspective and attitude.  One of the most appealing aspects that I have found in working in a research lab is the sense of freedom that one gets.  Eventually getting paid to do whatever it is that interests me seems like a very enticing career path.  I suppose that this sense of freedom is also a reason for me applying for a research fellowship.  Ideally, I would like to have some leeway when it comes to pursuing a research topic.  It has been brought to my attention that without a fellowship, graduate students are required to stay within the bounds outlined by the grants that support their research.  I believe that not having to worry about this will promote a more conducive environment for transformative research.
\newline
\tab One of the best reasons that I have for pursuing a PhD program is that I genuinely love computer science and am excited to continue my education.  I was introduced to computer science in high school when I took an Advanced Placement programming in Java class.  I enjoyed the class very much and successfully completed the advanced placement test.  After getting to college, my interest in computer science began to grow and still continues to evolve.  In the recent year I have been substantially more involved in the computer science program on campus than previously.  I spend almost all of my spare time working on research projects and other computer science related projects that are not necessarily associated with classes.  For example, I currently am taking a free online class that is hosted by Stanford University on functional programming, just for the fun of it.  I've also been invited to attend what they call the programming languages seminar here at the University of Minnesota.  We meet once a week to read and discuss research papers relevant to the programming languages discipline.  This has been an amazing experience, mainly because I am exposed to other types of research in the field that I am not currently familiar with. Also, I am able to network with other professors in the field of programming languages, and meet with graduate students and see what they are working on. \newline
\tab Another experience that had a large influence on me looking to pursue a PhD was the work that I did on extensible programming languages with Professor Eric Van Wyk.  Two semesters ago, I took a software engineering class that was taught by Professor Van Wyk.  Our course project was to build a language translator that translated some prototype language into a more general purpose language, in this case C.  I enjoyed working on this project very much, and at the end of the semester I was offered a position to work with Professor Van Wyk and the rest of the members of his research group.  I feel very thankful for being given this opportunity, and believe that if it were not for this experience, I would be in a very different place in my life.  Professor Van Wyk has served as a mentor to me, and an invaluable resource.   I hope that he continues to involve undergrads in his research, because it has had such a huge impact on my decision to pursue graduate school.    Over the last ten or so months I have thoroughly enjoyed my time working with him on the implementation of programming languages.  I find this to be by far the most interesting topic in computer science that I have yet to come across, and I am very interested in pursuing this topic in the years to come in graduate school.\newline
\tab A second research opportunity that has been given to me is a project in the field of data mining.  This experience has been much more fast paced, and I feel as though it has helped to prepare me for what it is going to be like to compile results and present my research.  Another perspective that this project has given me is carrying out research in a group setting.  With this project, I am working closely with a graduate student, and we collaborate with his advisor and one of the other research scientists in the lab on a regular basis.  In addition to this, we have also been collaborating with a professor from MIT, and I have been lucky enough to be included on the conference calls that we have with him every week or two.\newline
\tab One final experience that has had an influence on me applying to a PhD program is the volunteer work that I have done at the Hennepin County Library.  At the library, I tutor adults who are working towards their GED, or are taking Adult Basic Education classes.  The majority of the tutoring is done in a one on one situation, and the bulk of our clients are people who are new to the country, and have limited English speaking abilities.  For the most part, I tutor people in math, because it is my biggest strength, however, I also try to help people with English as well.  Teaching is never something that I would have previously considered, however, tutoring at the library has proved to be a very fulfilling experience.  Last semester I was required to volunteer there for a class that I was taking, but I enjoyed volunteering there so much that I continued to tutor there over the summer, and am still volunteering there today.  \newline
\tab My ultimate goal after pursuing graduate studies is to end up working somewhere in research.  I find the freedom that one gets in research to be very appealing.  Teaching is something that I feel may also interest me in the future.  I feel very fortunate to have worked with everyone that I have had the pleasure to do research with thus far, and to be able to give someone a similar opportunity one day would be very satisfying.

\end{document}










