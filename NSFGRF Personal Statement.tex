\documentclass[a4paper,12pt]{article}
\usepackage[utf8x]{inputenc}
\usepackage{graphicx}
\usepackage{subfig}
\usepackage[margin=1in]{geometry}
\usepackage[countmax]{subfloat}
\newcommand{\tab}{\hspace*{2em}}
%opening
\title{Personal Statement NSFGRF}
\author{Matthew Le}

\begin{document}

\maketitle

I genuinely love computer science. Unlike many of my peers, my love for the discipline was not at first sight, nor is something that came easily to me. When I was in high school, I took an advanced placement programming in Java class. I enjoyed the class very much and successfully completed the advanced placement test. After getting to college, however, I found that computer science classes at the collegiate level were much more difficult than what I had previously been exposed to. After being humbled by my first two college CS classes, I went back and retook the two introductory programming classes that I had originally tested out of. Since then, my interest in computer science has grown immensely, and has evolved into a love for the subject. By my senior year, I spend almost all of my spare time working on research and other computer science-related projects that are not necessarily associated with classes. For example, I currently am taking a free online class that is hosted by Stanford University on functional programming, just out of curiosity. The class has been instructional and allows me to interact with fellow computer scientists around the world. Through my involvement in the department, I've also been invited to attend ``the programming languages seminar'' a weekly journal club discussing the state-of-the-art programming languages research. This has been an amazing experience, because I am exposed to other lines of research in the field that I am unfamiliar with. Also, I am able to network with other professors in the field of programming languages, and meet with graduate students to learn about their research projects and challenges. I feel honored to be apart of the group, as I am currently the only undergraduate participating.  \newline
\tab Pursuing a PhD is something that I hadn't really considered until about a year ago. Conventional wisdom dictates that it is  more practical to go straight into the workforce after college as a software developer. However, after working in two different research labs, I have a very different perspective and attitude towards my career. One of the most appealing aspects that I have found in working in a research lab is the sense of intellectual freedom.  Eventually getting paid to do whatever I love is an enticing career path. This sense of freedom is also a reason for me applying for a research fellowship. I would like to have some leeway when pursuing a research topic, and shape my own research agenda. It has been brought to my attention that without a fellowship, graduate students are required to stay within the bounds outlined by the grants that support their advisors. I believe that an NSFGRF would create a more conducive environment for transformative research within the field of programming languages.
\newline
\tab Another experience that had a large influence on me looking to pursue a Ph.D. was the work that I did on extensible programming languages with Professor Eric Van Wyk. Three semesters ago, I took a software engineering class that was taught by Professor Van Wyk. I enjoyed working on this project very much, and at the end of the semester I was offered a position to work with his research group. I am thankful for this opportunity, and believe that if it were not for this experience, I would be in a very different place in my life. Professor Van Wyk has served as a mentor, and an invaluable intellectual resource. I hope that he continues to involve undergraduate students in his research, because it has had such a positive impact on my decision to pursue graduate school. Over the last ten or so months I have thoroughly enjoyed my time working with him on the implementation of programming languages. I find programming languages interesting because they provide us with the only means for communicating with machines. Additionally, I feel that there is a lot that a compiler can do with the information it gleans from a program during compilation, and I am excited to explore this further in graduate school.\newline
\tab A second research opportunity that has been given to me is a project in the field of data mining.  This experience has been much more fast paced, and I feel as though it has helped to prepare me for what it is going to be like to compile results and present my research.  Another perspective that this project has given me is carrying out research in a group setting.  With this project, I am working closely with a graduate student, and we collaborate with his advisor and one of the other research scientists in the lab on a regular basis.  In addition to this, we have also been collaborating with a professor from the Massachusetts Institute of Technology (MIT), and I have been lucky enough to be included on the conference calls that we have with him every week or two. This research experience was unique in exposing me to exploratory research where we propose hypotheses, test them empirically and draw a conclusion based on the results. Although, more often than not our hypotheses prove to be incorrect, I have grown to enjoy this type of research because it requires that we be creative in designing new experiments, something that I believe is crucial to performing transformative research.\newline
\tab One final experience that has had an influence on me applying to a Ph.D. program is the volunteer work that I have done at the Hennepin County Library in Minneapolis. I tutor adults who are working towards their GED, or are taking Adult Basic Education classes.  The majority of the tutoring is done in a one on one situation, and the bulk of the beneficiaries are newly arrived immigrants with limited English-speaking abilities.  For the most part, I tutor people in math, because it is my biggest strength, however, I also try to help people with English as well.Teaching was never something that I would have previously considered, however, tutoring at the library has proved to be a very fulfilling experience.  Last semester I was required to volunteer there for a class that I was taking, but I enjoyed volunteering there so much that I continued to tutor there today.  \newline
\tab My ultimate goal after pursuing graduate studies is to end up working somewhere in research.  I find the freedom that one gets in research to be very appealing.  Teaching is something that I feel may also interest me in the future.  I feel very fortunate to have worked with everyone that I have had the pleasure to do research with thus far, and to be able to give someone a similar opportunity one day would be very satisfying.

\end{document}










