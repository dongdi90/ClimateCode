\documentclass[a4paper,12pt]{article}
\usepackage[utf8x]{inputenc}
\usepackage{graphicx}
\usepackage[margin=.9in]{geometry}
\usepackage{subfig}
\usepackage[countmax]{subfloat}
\newcommand{\tab}{\hspace*{2em}}
%opening

\title{Previous Research Experience NSFGRF \vspace{-1ex}}
\author{Matthew Le\vspace{-5ex}}

\begin{document}

\maketitle

I am currently working on two research projects, both of which are funded by the NSF Research Experience for Undergraduates program.  The first project deals with 

I am currently working under an NSF REU on a research project dealing with finding modular and composable ways to design programming languages and extensions, under the guidance of Professor Eric Van Wyk.  The project that I am working on addresses this issue by creating a prototype host language, and then adding extensions to it.  The extension that I am working on adds support for matrices and the operations that accompany them.  Domain specific languages have become rather popular in the recent past because they allow the programmer to write code at a high level of abstraction, while enjoying the benefits of domain specific optimizations. 

 The downside to this is that you lose the functionality that you get with a general purpose programming language such as C/C++ or Java.  This research effort aims to add domain specific features to a host language, while maintaining the advantages that one gets with a standalone domain specific language. \newline
\tab We are writing these language extensions in a language called Silver, which was developed by Professor Van Wyk and some of his graduate students.  Silver is a parser generator, which takes in a language specification and outputs a compiler.  Silver supports a feature known as ``forwarding'' , which is a notion that was also coined by Professor Van Wyk, which facilitates modular language design.  The idea is that extensions to the host language can be reduced to constructs that already exist in the host language.  The extension that I have been working on makes heave use of forwarding, such that this extension is a ``pure'' extension, meaning that it is completely reduceble to constructs existing in the host language.  The benefit to this is that if some of the implementation of the host language gets modified, then the extension does not need to be changed.  \newline
\tab The first step to this project was to implement a simple host language.  This was already done for us by some of Professor Van Wyk's graduate students.  The language is a simple imperative language that generates C code.  The next thing that I did was design the underlying representation of matrices in C.  Since we are supporting multidimensional matrices, we needed some scheme that allows us to generalize to any number of dimensions.  We decided to use a C structure, which contains an array holding the sizes of each dimension, and a one dimensional array which contains the data of the matrix.  We then index a matrix using row-major indexing, similar to that used for two-dimensional arrays in C.  Now that we had our underlying representation figured out, it was time to start implementing the language extension using Silver.  \newline
\tab We began by first creating an additional type in the host language type system, which then forwards to a host type, in this case we forward to the host struct type.  We then created syntax and abstract productions for declaring matrices.  Matrix declarations are really no different than declaring any other variable, we simply need to add the name of the matrix being declared to the symbol table.  Another necessity that we implemented was matrix initialization.  This production forwards to a sequence of assignment statements, where we allocate space for the structure, the dimension sizes array, and the data array.  A few other productions that we added to the core matrix extension include matrix references, matrix arithmetic, and printing of matrices.  \newline
\tab After creating an extension that simply introduces this new data structure, we started adding some of the domain specific features that are found in MATLAB.  Such additional functions include logical indexing, referencing multiple elements of a matrix at a time, assigning to multiple elements of a matrix, and matrix permutations.  We ended up putting this functionality into an extension of its own, showing that not only host languages can be extended in a modular way, but extensions can also be further extended in a modular way.  These additional features have presented many opportunities for generating parallel C code, which is something that we are currently exploring now.\newline
\tab A second research opportunity that I am currently working on is another NSF REU funded project in the field of climate data mining.  The El-Ni\~{n}o Southern Oscillation (ENSO) is a phenomenon where Pacific Ocean warming has been linked to Atlantic hurricane activity\cite{enso}.  The problem is that the correlation between the two is quite poor.  This research effort aims to further investigate this relationship.  \newline
\tab When I began working on this project, the graduate student who I'm working with had already developed a method that explained the Pacific-Atlantic relationship more accurately than ENSO.  The idea behind it is that instead of monitoring the temperature of the Pacific Ocean, you monitor the location of the warming, i.e. what subset of the Pacific is the warmest.  This index correlated with hurricane counts in the Atlantic quite a bit better than the ENSO index.  However, the professor seemed reluctant to move forward with this method.  In the meantime, I had been experimenting with some variations of this index.  Many of these variations involved incorporating additional variables in addition to sea surface temperature.  After a few months, I had come up with an index of my own that performed a fair amount better than the previous one.  I had found that if I were to apply this spatial monitoring idea to different variables, I could come up with multiple separate indices, and then by taking their z-score and summing them together, we had one index that encapsulated the information of a few different variables.  This index has remained our top performing predictor of Atlantic hurricane activity when we restrict ourselves to Pacific Ocean information.  Over the past few months we have been compiling our results and testing the validity of this index.  I anticipate that it won't be long until we are able to publish these results.




%The research group deals with using data mining and machine learning techniques to further understand the causes and effects of climate change.  The particular project that I have been working on deals with discovering the effects of Pacific Ocean warming on Atlantic Ocean hurricane activity.  When I first began working on the project, they already had developed an algorithm that tracks the location of the region that has the warmest sea surface temperature, and used these metrics to predict hurricane activity in the Atlantic.  They found that if the location of the warmest region of the Pacific is in the western portion of the ocean, then there tends to be more hurricane activity than if the warmest region is in the eastern portion of the ocean.  The correlations of this new index proved to be much more accurate in predicting hurricane activity than traditional ENSO indices, which simply look at the sea surface temperature of a static region in the Pacific\cite{enso}.\newline
%\tab After a few months of working on the project, I developed a new index, which incorporates other variables into the mix, but still maintains this idea of monitoring spatial distributions.  I began working on this new index when I noticed that other indices that also tried to predict hurricanes used other variables such as sea level pressure and outgoing long wave radiation.  I tried using the same method that we used in the previous index on some of these new variables, as well as simply taking the magnitude of the variables within various regions.  I got some decent results, but nothing substantial.  After a while I attempted to combine these various indices into one index using various linear combinations, however, there was always this problem that I was trying to combine variables that had different units, rendering my results as nonsense.  I eventually realized that if I were to take the z-score of these indices, I could put everything into relative terms.  This approach ended up working very well, and the results that we saw from this were much better than the results that we were getting previously. We are currently in the process of compiling our results and preparing this work for publication. \newline

\tab In addition to conducting research, I have also had a few experiences with presenting my results.  About four months into working on my programming languages research project, we presented at the University of Minnesota Undergraduate Research Symposium.  this was an especially interesting experience because we were not just presenting to computer scientists.  It proved to be a rather difficult task to explain what we were working on to people who had never been exposed to computer science before. I also had the opportunity to present some of our work in the data mining research lab at the Expeditions in Computing Workshop hosted by the University of Minnesota. \newline
\tab I have had a great time working on these two projects, and feel that they have stimulated my interest in conducting research in the future.  I certainly have a different perspective on conducting research now that I have had some exposure to it, and feel that I am much more prepared than previously in carrying out research at the graduate level.  
\begin{thebibliography}{9} \vspace{-1ex}
%\bibitem{forwarding} E. Van Wyk, O. de Moor, K. Backhouse, P. Kwiatkowski.  Forwarding in Attribute Grammars for Modular Language Design.  In \textit{Proceedings of International Conference on Compiler Construction, Springer Verlag Lecture notes in Computer Science volume 2304}
%\bibitem{deforestation} Philip Wadler.  Deforestation: Transforming Programs to Eliminate Trees.  In \textit{Theoretical Computer Science}, Volume 73 Issue 2, June 22, 1990, Pages 231-248
\bibitem{enso}M.C. Bove, J.J. O'Brien, J.B. Elsner, C.W. Landsea, X. Niu.  Effect of El Nino on U.S. Landfalling Hurricanes, Revisted.  \textit{Bulletin of the American Meteorological Society}, Volume 79, Number 11
\end{thebibliography}
\end{document}










