\section{Appendix - Work in Progress}
\subsection{Increased seasonal predictability through monitoring the SST warming patterns and associated impact}
In addition to monitoring the location of the largest warming anomaly in the Pacific, we have also monitored the resulting deep convection and other spatial patterns such as the mean SST empirical orthogonal function (EOF). A combination of such quantities may yield a significant improvement over the state-of-the-art statistical forecasting algorithms.
We have already done an extensive analysis of this multi-variable index and have seen some improvements in performance. We are currently working on identifying the sources of accuracy for such an index.

\subsection{Monitoring the spatial warming patterns in the Pacific allows us to by-pass the ENSO predictability barrier}
While S-ENSO is more robust than tradition NINO indices to increased lead times, we are investigating how predictable such spatial patterns are. If we are able to predict the warming distribution several months in advance, then that would be a significant contribution to TC forecasting techniques. 
We have compiled results studying the evolution of both our S-ENSO index and EOF indices. There seems to be a predictable pattern that can be resolves using statistical methods. We are also investigating whether current SST forecast models such as ECMWF are able to reproduce S-ENSO.

\subsection{Monitoring the spatial distribution of the warmest and coldest SST anomaly regions in the Pacific encapsulates the Pacific SST EOF}
We also built an index that monitors the distance between the coldest and warmest SST region in the Pacific, which is similar to what the EOF does in terms of looking at extremes to explain variability. Such an analysis allows for a more detailed monitoring of ENSO's evolution. Our preliminary analysis shows that both the spatial distribution and the EOF's first principal component explain the same amount of TC variability.

\newpage