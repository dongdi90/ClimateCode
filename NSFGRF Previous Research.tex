\documentclass[a4paper,12pt]{article}
\usepackage[utf8x]{inputenc}
\usepackage{graphicx}
\usepackage{subfig}
\usepackage[countmax]{subfloat}
\newcommand{\tab}{\hspace*{2em}}
%opening
\title{Previous Research Experience NSFGRF}
\author{Matthew Le}

\begin{document}

\maketitle
In the past ten months or so, I have had two major exposures to research in the field of computer science.  The first of which has been on a project working for Professor Eric Van Wyk, who's research interests are in both programming languages and software engineering.  The project that we have been working on deals with extensible programming languages, and finding efficient ways to extend programming languages.  Professor Van Wyk and his graduate students have been developing a host language that is similar to the C programming language, and my role has been to add an extension to this language.  The extensions that I have been working on in particular deals with adding matrix data structures and the operations that accompany them to the host language.  This research opportunity in particular has shaped my interests in computer science the most.  I hope to pursue research in programming languages in graduate school, and feel as though I already have a fair amount of experience and exposure to the field  because of the work that I have done on this project


%In the past ten months or so, I have had two major exposures to research in the field of computer science.  The first of which has been on a project working for Professor Eric Van Wyk, who's research interests are in both programming languages and software engineering.  The project that we have been working on consists of developing a programming language that is written in a new innovative way, such that the language can easily be extended, this technique, created by Professor Van Wyk, is known as forwarding.  The basic structure of the project is that you create some host language, which serves as an example of a programming language in its initial release state.  Along with the host language, is a number of extensions to this language.  The idea is that each programming language has its own key features, and people use different languages to handle tasks within different domains, however, with this new approach to implementing programming languages, you could use one host language, and extend it to facilitate easier programming in various domains.  The host language is similar to that of a general purpose low level programming language such as C.  My role in the project has been to implement an extension to this language.  The extension that I have been working on adds support for matrices, and the typical operations that accompany such data structures.  For the most part, I have been doing the majority of the implementation of this extension by myself, but not without a wealth of advice from Professor Van Wyk and some of his graduate students.  We meet each week to discuss what we have been working on, and what issues we have run into.  With this aspect, I have learned a great deal about developing software as a team, and performing research in a group setting.  \newline
\tab I feel that working on this project has exposed me to many of the trials and tribulations of performing research in the field of computer science.  I have been told that you'll never accomplish what you set out to do on a research project on your first try, and if you do, you probably aren't researching anything interesting.  When I first began the project, we only supported two-dimensional matrices, just to keep things simple, and focused on implementing the core elements necessary to make this extension useful.  Well, after a few months we had a solid working extension, but decided that only supporting two-dimensional matrices was too restrictive, and essentially scrapped all of the code that I had written.  Many of the ideas and concepts that were developed remained the same, but the implementation changed drastically.  \newline
\tab Another aspect of performing research that I have been familiarized with through this experience is presenting our results.  About three or four months into the project there was an undergraduate research symposium for all undergraduates who were performing research in all types of fields.  This was a particularly interesting experience, because it is one thing to explain your work to someone who is reasonably familiar with what you are doing, and another thing to explain it to someone who has never heard of anything even remotely related to what you are working on.  The research symposium was set up so that each group of researchers had a place to hang a poster, and then people would walk by and ask you questions about your work.  This was an excellent experience, because I was able to meet professors who I had never gotten a chance to talk to face to face before, and tell them about what I was working on\newline
\tab For the past three to four months, I have also been involved in another research project in the field of data mining.  Our project deals with trying to better understand the relationship between Pacific Ocean warming and Atlantic hurricane activity.  When I first began working on this project I took the place of another undergraduate who left to take an internship opportunity.  The project was already very well developed, and it wasn't long until it was time to prepare our work for publication.  Recently, we have been putting the majority of our effort into compiling our results and sending them out to colleagues for their input.  I anticipate that it won't be long until our work is in a publishable state.  This has been a very interesting aspect of conducting research that I have not been exposed to with my other project.  It takes a lot more time and preparation to prepare something for publication than I previously thought.  I think that it has been a really good experience being involved in such a process, and I believe that it will help prepare me for what is to come in graduate school
\end{document}










